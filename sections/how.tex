\section{An Introduction to Bitcoin}
\anote{There are a few conflicting goals in this section. One is to introduce the basic terminology and concrete details about how the present Bitcoin works. Another is to introduce the key abstractions we'll use to understand Bitcoin variants. Another is to understand Bitcoin in context.}


\subsection{An Alternate History of Bitcoin in Steps}
\anote{The purpose of this section is to explain what's new about Bitcoin, and how it might have been derived systematically as an improvement to other approaches.}

%The goal is to have basic checking account functionality for Alice and Bob.

We will describe 5 hypothetical digital currency systems, each of which is a successively better approximation to Bitcoin. 

\paragraph{Step 1: Imagine there's a trusted third party (TTP)}
Rather than ``accounts,'' there are addresses that have balances. These addresses are the public keys of key pairs that anyone may generate; for any address, anyone who knows the corresponding private key has spending power over that address via digital signatures. We will defer the question of initial balances.

The TTP allows anyone to send it messages and gives all nodes a consistent view. If the owner of address $A$ wants to send address $B$ value $v$, she sends to the TTP a signed statement, called a transaction, saying ``address $A$ transfers value $v$ to address $B$.'' 

The TTP verifies each transaction (checks that the signature is valid and $v \leq \mbox{balance}_A$) and ignores the ones that don't verify. The TTP publishes a public log containing all transactions, in the order they occurred. (Note that this allows anyone to verify that the TTP is not inserting any invalid transactions into the log.) 


\paragraph{Step 2: Removing the TTP}

If the owner of address $A$ wants to send value $v$ to address $B$, she broadcasts a signed statement to all other nodes saying ``address $A$ transfers value $v$ to address $B$.''

At regular intervals (say 10 minutes), a distributed set of nodes (replacing the TTP) conduct a protocol to (a) propose a batch of valid transactions, called a block, and (b) vote on these proposals to select one. 

Well known voting protocols have been studied, under some standard assumptions. First, it is assumed that a majority of the nodes are honest. Second, it is assumed that honest nodes are able to broadcast synchronously to every other node. Dishonest nodes, on the other hand, are assumed to be able to equivocate, sending one message to some nodes and a different message (or nothing) to others. \footnote{For efficiency, there are randomized protocols.\anote{survey of randomized consensus, breaking $n^2$ barrier}}

Underlying all this is the fundamental assumption that there is a fixed set of nodes that is known to everyone. (The identities of nodes in the consensus protocol need not be related in any way to ownership of addresses used to transact quantities of the currency.) In the next step we will weaken this assumption. 

\paragraph{Step 3. Minimizing assumptions about nodes}

In reality Bitcoin nodes can come and go as they please, and are not expected to identify themselves. So we replace the assumption that there is a fixed, known set of nodes with a much weaker one: Imagine that there’s a magic token that lands on a random node (chosen uniformly) at regular intervals \footnote{The magic token is recognizable and unforgeable: if a node receives a message from a node holding the token, it knows that the sender did in fact hold the token.}

As before, nodes broadcast each transaction to all other nodes. When a node gets the token it collects all the transactions it has seen (in the most recent token interval), verifies them, and broadcasts the block. Nodes cannot broadcast blocks without possession of the token.


\anote{Description of the hash chain data structure, how participants process blocks.}

 
Each block contains a pointer to the previous block in the chain. In the best case, all of the blocks will form a single chain, but in practice the ``chain" may be a tree because multiple blocks might point to the same predecessor block.  When an honest node makes a block, it will extend the longest branch of blocks that it sees containing only valid transactions (``longest-branch rule''). \footnote{Different nodes may see slightly different trees of blocks, since the network is not reliable.}
 

We assume that more than half of the nodes are honest (meaning they execute the prescribed protocol correctly). Therefore in each interval the token selects an honest node with probability greater than $\frac{1}{2}$. As a result, assuming that network inconsistencies between honest nodes are small, in the long run a 
valid branch will end up longer than any chain containing an invalid transaction, with probability one. 

\paragraph{Step 4: Incentives}


To better justify the assumption that a majority of nodes follow the protocol, we provide an incentive for such behavior. 
\footnote{Everything described so far is merely an abstract system for transferring and tracking balances. For the incentive system to work, these balances must be interpreted as money.}
When a node makes a new block, it gets to include in that block a special transactions that pays a ``block reward'' to an address of its choosing (intuitively, to ``itself,'' although there is no protocol-level correspondence between nodes and addresses).  The block reward is currently 25 BTC.\footnote{This is also the mechanism by which new Bitcoins are distributed, in an arguably fair way.}
 

This creates an incentive to create a block that will end up on the long-term consensus branch, to make sure that the block reward transaction is considered ``real'' by consensus. If the majority of nodes are honest (and therefore follow the longest branch rule), this creates an incentive for all nodes to adopt the longest branch strategy, and not to include invalid transactions in the block. 



\paragraph{Step 5: Removing the token: proof-of-work}

Nodes  compete  for the right to broadcast a block by  racing  to  solve  a  proof-of-work  based  on  hashing.  Proof-of-work replaces the token, and has the property that nodes acquire the token with probability proportional to the fraction of the network's total hash power that they control.

Now, instead of the majority of nodes being honest, we require that the majority of hash power be controlled by honest nodes. Except for that change, the argument for consensus in Step 3 holds here as well.


This suggests a bootstrapping argument. If the transaction log is secure (has integrity and availability), then it will be useful as money, and therefore the block reward makes a worthwhile incentive. This in turn will encourage a large amount of participation, which ensures the transaction log is secure.

This appears to be a virtuous cycle. On the other hand, neither system can exist without the other. If the integrity or the availability of the transaction log can be easily attacked, then we cannot expect it to maintain much value as a currency. Likewise, if the currency is not valuable as money, then it will not be an effective incentive for encouraging sufficient participation to prevent successful attacks.


\subsection{Law, Log, and Index: An Abstraction of Bitcoin as a Platform}

Abstractly speaking, what functionality does Bitcoin -- or more importantly, the innovations embodied by it -- provide? Originally presented as a mechanism for online ``payments,'' in the abstract it is a much more versatile platform.

\begin{enumerate}
\item (Log) A sequential broadcast medium. Every user eventually (in a short amount of time, with high probability) agrees on a global sequential log of messages (and every user receives every message in its entirety). Any user can publish a message, and it will be included in the log within a short time (although, a monetary fee may be required for timely service).
\item (Index) Participants on the network maintain a globally replicated storage index. Every message in the broadcast log is processed, and in some cases causes updates to the index. These indexes can be relied on as concise summaries of the public broadcast record.
\item (Law) The broadcast log may additionally contain data that reflecting the result of computations, including queries to the index.
\end{enumerate}

This framework can be used to understand the design differences between many Bitcoin-related proposals, all of which are special cases of the general structure. As an example, the standard usage of Bitcoin can be described as the following instance:
\begin{enumerate}
\item The broadcast log is used to publish signed messages called ``transactions'', which reflect an intent to transfer quantities of currency from one account to another. Quantities of the currency may also be set aside as fees.
\item Participants maintain a \emph{ledger} containing a mapping between every quantity of currency and the public key of its current owner. This index must be updated to reflect the new balance after every transaction.
\item Transactions are considered ``valid'' only if the attached signatures are valid with respect to the public keys in the current ledger, and if they spend a reasonable amount of money. In fact, only transactions that are valid according to these rules may be included in the log (although arbitrary data --- including invalid transactions --- may still be published as auxiliary data contained within a valid transaction).
\end{enumerate}
All three of these components are necessary, and complement each other. The sequential log is used to ensure that no two transactions double-spend the same quantity of currency. On the other hand, since only valid transactions carry fees, the index is used to ensure that data needed to validate transactions is readily available.

Since only valid transactions are included in the log, end-users (such as merchants with mobile devices) need only check that a transaction is included in order to confirm a payment. Although the log necessarily grows without bound, the mutable index does not; therefore nodes can ``prune'' old data from their main disks. These optimizations are discussed in more detail later on.

Most proposals for Bitcoin-like variations, including overlay currencies such as Mastercoin and Colored Coins as well as more ambitious altcoins like Ethereum, can be described in a similar way as we will show in Section~anote{refer}. Additionally, these key components provide a basis for evaluating the resource complexity of various schemes and use-cases.

The view of Bitcoin as a platform is that the currency is only inherently useful to the extent that it allows you to pay for transaction fees. In this sense it is comparable to postage stamps.


\subsection{System Attack Model and Consensus Requirements}

The Bitcoin mining consensus proof from the whitepaper, assuming a majority of the hash power is ``honest''. Bitcoin's solution to the consensus problem is novel, although the proof known so far is for the wrong model (honest rather than altruistic). Given a set of rules, in a distributed system, the problem remains of how to choose a correct order of operations.

Although, so far, the economic incentive system seems to work, there are only heuristics and no clear model. As best we can tell, the reasoning is circular. Incentive system is needed to make the system secure. The built-in currency that rewards participants has to be valuable for the incentive to be meaningful. The currency is valuable only if the system is secure.\footnote{For example, when there have been even temporary lapses in service or bugs announced, the price of Bitcoin has fallen~\cite{bitcoin-fork-news}.} Attacks are discussed further in Section\anote{refer}.

Bitcoin's assumptions are weaker than in a standard distributed system - most notably, there are no pre-established identities. On the other hand, the assumptions are about the rational preferences of population - specifically that they respond to incentives, even when the incentives are inconclusive. Seems to be circular?

Stabilizing consistency is weaker than a typical distributed system. In a typical distributed system, you receive an acknowledgment at some finite time that the transaction has been committed. In Bitcoin, there is no such acknowledgment, and users must user their own discretion about how long to wait to consider a transaction committed. \anote{This is partially a historical note, meant to clarify how Bitcoin's version of consensus/broadcast relates to more standard notions}

A typical definition of fairness or liveness would specify that *any* transaction should eventually accepted, without reference to adequate payment. However, since Bitcoin operates in a model with no established identities, this would be impossible due to the ability for an attacker to create sybil identities and flood the system.

\subsection{Concrete Details}
- Bitcoin's underlying mechanism.

- Mining, transaction propagation.

- Introduction of currency. Use of exchanges, market places. Financial ecosystem.

\subsection{How Changes are Applied}

- Discussion of governance and the need for social out-of-band consensus to agree on rule changes. Soft forks, hard forks, and policy. Very few hard forks in history.

- Discussion on the forums and mailing list that provide.

- Ability for altcoins to develop.

\subsection{Towards an Economic Model}

- Pooled mining and infrastructure investment.

- A history of pooled mining, and what we can infer empirically about the motivations of miners? Prefer low variance.

